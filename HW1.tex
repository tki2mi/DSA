\documentclass[12pt]{article}
\usepackage[margin=1.5cm]{geometry}
\usepackage{xurl}
\usepackage{algorithmicx}
\usepackage{algpseudocode}
\usepackage{xeCJK}
\setCJKmainfont{NotoSerifTC-Regular.otf}
\title{Data Structure and Algorithm HW1}
\author{R11522709 機械所\, 碩一\, 石翊鵬}
\begin{document}
\maketitle
\section*{1. What if you became a DSA TA?}
\subsection*{Proof of conjecture 1}
Consider $f(n)\geq 0$ and $g(n) \geq 0$ that $\lim_{n \to \infty} {\frac{f(n)}{g(n)}} = c$, where $c \in R$\newline
To prove $\lim_{n \to \infty} {\frac{f(n)}{g(n)}} \geq 0$ for all $n \in R$, we can consider h(n) = 0\newline
Since \(f(n) \geq 0\) and \(g(n) \geq 0 \) for all \(n\in R\), \(\frac{f(n)}{g(n)} \geq h(n)\) for all \(n\)\newline
therefore,
\[\lim_{n \to \infty}{h(n)}\geq \lim_{n \to \infty} {\frac{f(n)}{g(n)}}\]
\[\lim_{n \to \infty}{h(n)} = 0 \Rightarrow \lim_{n \to \infty} {\frac{f(n)}{g(n)}}=c \geq 0\]
By difinition of limit, there exists $\epsilon > 0$, $n_0 > 0$ such that for all $n > n_0$, \newline
\[|\frac{f(n)}{g(n)}-c|<\epsilon\]
\[-\epsilon < \frac{f(n)}{g(n)} -c < \epsilon\]
\[-\epsilon + c < \frac{f(n)}{g(n)} < \epsilon + c \]
In case that $0 < c < \epsilon$
\[-\epsilon + c < 0 < \frac{f(n)}{g(n)} < \epsilon + c \]
In case that $c \geq \epsilon$
\[0 \leq -\epsilon + c < \frac{f(n)}{g(n)} < \epsilon + c \]
\[0 \leq f(n) < (\epsilon + c)g(n)\]
Let $(\epsilon + c) = c_1$
\[0 \leq f(n) \leq c_1g(n)\]
which shows $f(n) = O(g(n))$, QED
\subsection*{1.1 - all by myself}
Consider $c_1, c_2 >0$\
for all $n > 0$\newline
if
\[\frac{c_1}{c_2}n \geq 1\]
then
\[c_1n^3\geq c_2n^2\]
let
\[n_0 = \frac{c_2}{c_1}\]
\[\frac{c_1}{c_2}n_0 = 1\]
consider $k > 0$
\[\frac{c_1}{c_2}(n_0+k) = 1+\frac{c_1}{c_2}k\]
where
\[\frac{c_1}{c_2}k > 0\]
that is, for all $n \geq n_0$,
\[0\leq c_2n^2 \leq c_1n^3\]
therefore, if
\[0 \leq f(x) \leq c_3n^2\]
for all $n > n_1$ where $n_1, c_3 > 0$\newline
then
\[0 \leq f(n) \leq c_3n^2 \leq \frac{c_3c_1}{c_2}n^3\]
for all $n > max(n_0, n_1)$, QED
\subsection*{1-2 - all by myself}
Let $t_m = the\ number\ of\ while\ checks$, $k = the\ index\ of\ key$, $d_n = time cost of each line to execute once$\newline
Since the value of m is assigned to l in the beginning of each while loop, m will start from 1 and increase by 1 when while loop is executed once, moreover, $A[m] \leq key$ will always be satisfied.\newline
therefore,
\[t_m = k\]
in worst case, key is not in the array, in this case
\[T(n) = d_1n + d_2(n-1) + d_3(n-1) + d_4 * 0 + d_5(n-1) + d_6 * 0 + d_7(n-1) + d_8(n-1) + d_9\]
\[T(n) = c_1n + c_2\]
where $c_1$, $c_2 > 0$
\[\lim_{n \to \infty} {\frac{c_1n + c_2}{n}} = c_1\]
By conjecture 1, time complexity of the algorithm is $O(n)$

\subsection*{1-3}
ref: \url{https://math.stackexchange.com/questions/925053/using-limits-to-determine-big-o-big-omega-and-big-theta#comment6149810_925053}\newline
\[f(n)=\Theta(n^2)\Longleftrightarrow there\ exists\ positive\ (n_0, c_1, c_2)\ such\ that\ c_1n^2\leq f(n)\leq c_2n^2\ for\ all\ n\geq n_0\]
Let \(c_1 = 1\), \(c_2 = 10\) and \(f(n) = n^2(sin(n) + 2) + n\)\newline
for all \(n > 0\), \(n^2  \leq f(n) \leq 10n^2\)\newline
Suppose
\[\lim_{n \to \infty} {f(n)} = L\]
where $L \in R$, then there exist $\epsilon > 0$, $n > 0$ such that for all $n > n_0$,
\[|f(n)-L|<\epsilon\]
Let $f(n_1) = \epsilon_1$ where $n_1 > n_0$ and $\epsilon_1 < \epsilon$
\[f(n_1 + 2\pi k) - f(n_1) = ((n_1+2\pi k)^2-n_1^2)(sin(n_1) + 2) + 2\pi k\]
where $k \in n$
\[\lim_{k \to \infty}{f(n_1 + 2\pi k) - f(n_1)} \neq 0\]
\[\Rightarrow |f(n+2\pi k) - L| > \epsilon\]
for some $k > k_0$  where $k_0 \in n$\newline
That is, $\lim_{n \to \infty} {\frac{f(n)}{n^2}} $does not exists, therefore, the proposition is WRONG.

\subsection*{1-4 - all by myself}
\[lg(n) = 2lg(\sqrt{n}) = 2\frac{ln(\sqrt{n})}{\ln(2)}\]
Consider
\[\lim_{n \to \infty}{\frac{lg(n)}{\sqrt{n}}} = \lim_{n \to \infty}{\frac{2}{ln(2)}\frac{ln(\sqrt{n})}{\sqrt{n}}}\]
Let \(\sqrt{n} = x\)
\[\lim_{n \to \infty}{\frac{2}{ln(2)}\frac{ln(\sqrt{n})}{\sqrt{n}}} = \lim_{x \to \infty}{\frac{2}{ln(2)}\frac{ln(x)}{x}}\]
By L'Hopital's rule
\[\lim_{x \to \infty}{\frac{2}{ln(2)}\frac{ln(x)}{x}} = \lim_{x \to \infty}{\frac{2}{ln(2)}\frac{\frac{d}{dx}ln(x)}{\frac{d}{dx}x}} = \lim_{x \to \infty}{\frac{2}{ln(2)}\frac{\frac{1}{x}}{1}}=0 \in R\]
By conjecture 1, $lg(n) = O(\sqrt{n})$, QED

\subsection*{1-5 - all by myself}
Consider
\[\frac{\lim_{n \to \infty}{\Sigma_{i = 1}^{n}i^n}}{\lim_{n \to \infty}{\Sigma_{i = 1}^{n}n^n}} = \lim_{n \to \infty}{\Sigma_{i = 1}^{n}(\frac{i}{n})^n}\]
and
\[\lim_{n \to \infty}{\int^n_1(\frac{i}{n})^n di}\]
and
\[\lim_{n \to \infty}{\int^{n - 1}_0(\frac{i}{n})^n di}\]
By definition of $\Sigma$,
\[\lim_{n \to \infty}{\int^{n - 1}_0(\frac{i}{n})^n di}<\lim_{n \to \infty}{\Sigma_{i = 1}^{n}(\frac{i}{n})^n}<\lim_{n \to \infty}{\int^n_1(\frac{i}{n})^n di}\]
\[\lim_{n \to \infty}{\int^n_1(\frac{i}{n})^n di} = \lim_{n \to \infty}{\frac{1}{n^n}\int^n_1 i^n di} = \lim_{n \to \infty}{\frac{1}{n^n}\ \left[i^n\right]^n_1 }=\lim_{n \to \infty}{\frac{n^n}{n^n}}-\lim_{n \to \infty}{\frac{1^n}{n^n}}=1-0=1\]
\[\lim_{n \to \infty}{\int^{n-1}_0(\frac{i}{n})^n di} = \lim_{n \to \infty}{\frac{1}{n^n}\int^{n-1}_0 i^n di} = \lim_{n \to \infty}{\frac{1}{n^n}\ \left[i^n\right]^{n-1}_0 }=\lim_{n \to \infty}{\frac{(n-1)^n}{n^n}}-\lim_{n \to \infty}{\frac{1^n}{n^n}}=1-0=1\]
By squeeze theorem,
\[\lim_{n \to \infty}{\Sigma_{i = 1}^{n}(\frac{i}{n})^n} = 1\in R\]
By conjecture 1, ${\Sigma_{i = 1}^{n}i^n} = O(n^n)$

\subsection*{1-6 - all by myself}
In second last line, since $c$ can be positive or negative, it is possible that $\frac{1}{2^c}>2^c$ and therefore $f(n)> \frac{1}{2^c}$\newline
The correct proof:
\[|lg(f(n))-lg(g(n))|=c\Rightarrow lg(\frac{f(n)}{g(n)}) = c \Rightarrow\frac{f(n)}{g(n)}=2^c\]
\[\Rightarrow |lg(f(n))-lg(g(n))|=c\Rightarrow lg(\frac{f(n)}{g(n)}) = c_1 \Rightarrow\frac{f(n)}{g(n)}=2^{c_1}\]
where $c_1 = \pm c$\newline
Take $c' = 2^c-1$, we have $f(n)\leq c'g(n)$ for all $n > n_0$, QED\newpage

\section*{2. DSA Judge}
\subsection*{2-1}
ref: chatGPT\newline
psuedo code:
\begin{algorithmic}
    \Procedure{FindLost}{$A, n$}
    \For{i from 0 to 1}
    \State $m= n/2$
    \State $l = 0$
    \State $r = n$
    \While{$(r - l)$ > $1$}
    \If{$A[m] < m + (2-k)$}
    \State{$l = m$}
    \Else
    \State{$r = m$}
    \EndIf
    \EndWhile
    \EndFor
    \State {$max(r_1, r_2) += 1$}
    \State\Return $r_1, r_2$
    \EndProcedure
\end{algorithmic}
time complexity is almost the same as binary sort in worst case, the code will be run $lg(n)$ times, the time complexity $T(n) = O(lg(n))$

\subsection*{2-2 - all by myself}
\begin{algorithmic}
    \Procedure{coupleDouble}{A,n}
    \State $isCouple = 0$
    \State $m = A[n]/2$
    \For{i from 0 to n}
    \If{$A[n-i] < m$}
    \State $m=m/2$
    \State $k = -k$
    \EndIf
    \State $isCouple += k$
    \EndFor
    \If{$isCouple == 0$}
    \State $return True$
    \Else
    \State $return False$
    \EndIf
    \EndProcedure
\end{algorithmic}

\end{document}